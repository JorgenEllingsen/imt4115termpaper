% !TEX encoding = UTF-8 Unicode
% !TEX root = thesis.tex
% !TEX spellcheck = en-US
%%=========================================
\chapter[Bot Metrics]{Do bots,  make click, text and behavior metrics irrelevant?}
\section{Management Summary}
In todays society, where everyone has a online presence, and many of our financial transactions are based on online content, one has to maneuver smartly through the grapevines which makes up the internet. In the article we find examples that many costumers buy based on recommendations, and reviews alone, and are therefore prone to being manipulated by fake content like bot metrics. In the 2014 Superbowl, people were in general tweeting more during the commercials than they were during the games. If bots also interact in this, we can safely say that these types of interactions can influence what we buy and why. 
It is not only through financial transactions bots have entered the fray, its also in society in general. In the 2016 U.S election, research shows that about 400 000 bots influenced the election on Twitter, giving rise to the claim that we may be influenced by non human entities, and that this might have had some deciding factor. The article includes research about how bots are perceived by humans, and shows that they found the Twitterbot in this case very trustworthy, which reinforces the thought that bots and their actions might implicate us in a larger manner than what was first believed. We discuss how the metrics influence us through direct examples and their implications. The article concludes that bots will always be a part of the metrics, because bot detection is a very complex problem which cannot guarantee a 100\% bot free environment. 

\section{Abstract}
This article will focus on bot metrics used in E-commerce(social media analytics), as well as bot metrics in other commercial contexts. According to research which will be presented in this article, bots are used in numerous ways, from rigging elections, hiding its own purpose by masquerading as users, to spreading misinformation from unverified sources, to promoting sales as well as working as sales representatives in e-commerce. This article then tries to delve into metrics by looking at social media statistics(SMA) in order to highlight why it may be worthwhile to fool the metrics and further uses a real world example in the case of IMDB reviews in order to give an understanding of how metrics in worst case could influence a buyers' decision, as well as how metrics in the U.S 2016 elections might have been a deciding factor for some twitter users, affecting the outcome of the election. The article is concluded by a discussion on whether or not bot metrics in behavior metrics cause metrics themselves to be irrelevant. 

\section{What are bots and how are they used?}\label{intro:howwhenwhy}
In general, bots are types of software which run automated tasks(which are repetitive and allows for the program to be faster at doing some tasks than what normal humans would be able to. There are many types of bots, for example gaming bots which can be programmed to efficiently react faster than what a human would be able to on certain events in the game, thus making the humans better players. Others examples include auction bots, which hunt for bargains. Ebay went to court in 2000( \cite{Computerworld:Ebay}) in order to stop this type of behavior, the federal courts in turn then decided to block Bidders Edge and their bots from accessing Ebays API.
\\

In the US 2010 mid-term election, social bots were reportedly used to influence the support of some candidates while effectively slandering others by tweeting thounsands of tweets which pointed to websites which contained fake news about the other candidate (\cite{ICWSM112850}). This example highlights the potential seriousness of social bots. \label{midtermElection}
\\
It is not only in elections these bots have a great influence, during the aftermath of the Boston bombings it was noted that social media(and bots) had a great effect on getting the message out there(\cite{Cassa:Twitter}), but there was also a negative effect which bots attributed on Twitter by re-tweeting peoples' unverified accusations or even checking out the credibility of the source, causing more hurt than good. \\
\\
Furthermore bots are used to mimic human activity, in for example service applications which appear to be human interaction but is instead a bot which for example tries to "help" you based on keywords. Bots are used in many areas, but are probably most known for their maliciousness in areas such as spam, bandwidth-thiefs, scrapers, worms and viruses as well as as nodes in a larger scale bot-nets is used for example for DDOS. These bots can be part of bigger bot-nets in order to for example generate revenue through for example click fraud. 
\\
\\
The observers article on fake traffic \cite{Observer:FakeTRAF} describes the problem and its extent: 
\begin{quote}
A recent study by comScore found that 54 percent of display ads shown in thousands of campaigns between May 2012 and February 2013 never appeared in front of a human being. Rather, the traffic came from bots. As an advertiser, this would be like buying a billboard you were told was seen by thousands of cars a day only to find out that was because the billboard sat next to the assembly line at a Ford plant. Sure, that’s a lot of cars, but there’s no one in them.
\\
\\
In fact, the system is so broken that, for some publishers, knowingly buying traffic that comes from bots is part of their business model. An anonymous publishing executive, who claimed to be buying up to \$35,000 worth of traffic per day, recently told Digiday that for publishers running an arbitrage model, all that matters is profit; quality of traffic does not factor into the equation.
\newline 
\mbox{} 
\hfill 
\cite{Observer:FakeTRAF}
\end{quote}

This quote from comScore highlights a important point in today's society. How does click metrics influence us, and how are they used? Are they useless, can they be used to measure something even though results most likely have been contaminated? and how does this type of behavior influence the buyers decision? These and some others are the question I will answer during this chapter. 


\section{What is  social media statistics, and how is it utilized?}\label{intro:howUtilized}
With the rise of social media use, social media marketing got a bigger foothold, allowing marketeers to interact with their buyers to a bigger degree, and it allowed the buyers to interact with their brands. Following this trend, many companies have big social media presences. Social media statistics allows the companies to closely monitor and narrow down their demographic and easily make specific content for a specific demographic. Furthermore, it allows for the company to get free publicity through what is called "organic reach" (getting exposure through users' activity in order to generate more clicks from users' contact) as well as for example paid reach(ads sent to a predefined demographic)
Social media statistics allows for all of this to happen, as \cite{Singh2013} mentions:
\begin{quotation}
Social Media statistics about audience likes and dislikes makes it plausible to employ "push marketing"-techniques to target audiences with advertisements that are relevant to their interests. Also, parameters such as click through rates or CTR can further quantify the success (or failure) of an advertising campaign. 
\end{quotation}
This type of push marketing, utilizes the information the users' themselves have provided to the mother service(e.g. Facebook), like for example metadata which gives a certain characteristic which allows them to be targeted by the broad filters which services like e.g. Facebook uses. 
\\
Businesses also utilizes what is called pull marketing, which is a little more subtle, e.g. referrals, social competitions("like and share" and get the \textit{chance} to win X") these methods can generate traction in social media, and be spread by word of mouth. 
\\
\\
But most notably, these types of metrics can be obfuscated by bots, masquerading as real users. These types of metrics can be abused by bots because bots are hard to detect using convetional statistics and other methods, in an article entitled "Fool Me If You Can: Mimicking Attacks and Anti-Attacks in Cyberspace" \cite{6601602} researchers establish that legitimate cyber behavior can be simulated, and further that it cannot be discriminate mimicking attacks from legitimate cyberevents, in the case of big bot nets. Furthermore, the article concludes that conventional staticial approaches such as viewing time interval and browsing length, can be easily mimicked if this data already exists. This all highlights the problem with bots.


\subsubsection*{An example of why metrics may be come obfuscated}\label{obfuscatedmetrics}
What happens to social media analytics in a business environment when bots enter the fray? Depending on the function of the bot, its behavior can vary greatly. If the bot is what is called a crawler, the bot will crawl through real peoples feed and try to mimic their behavior, and then crawl and gather more data from other people in the first peoples feed and so on. This type of activity has atleast three types of repercussions, it firstly generates fake interest for other companies' social media analytics, it hides their original intention(liking "this page") and it likes "this page" and generates fake revenue and publicity.
\\
\\
\label{Symantec:mockingbirds}A good real-life example is \cite{Symantec:Narang}, which had a set of inter chained accounts posing as real accounts in order to spread their own links about some dubious diet pills. The accounts stole meta data from real accounts. Instead of using compromised
accounts to tweet spam links, they were using accounts that impersonated brands and
celebrities. Symantec goes on to describe how they defined three types of accounts involved in this scam:
\begin{itemize}
\item {\textbf{Mockingbird}: Used real data from real celebrities for impersonating these individuals}
\item{\textbf{Parrot}: Fake accounts using stolen tweets and photographs of real women}
\item{\textbf{Egg}: New users with no set avatar}
\end{itemize}
Mockingbirds have the goal of promoting the weight loss tricks. These mockingbird-tweets would get thousands of likes from Parrot-accounts which spiders through the real accounts of the mockingbirds. Parrots then follow any and everyone in the hope that users will follow them back because they are using avatars of attractive women, a tactic that has proven very efficient. The Parrots have real content that they post each day which is fake, an not only the content of the Mockingbird, in order to seem more real. These tweets are usually stolen from real accounts in order to seem real.  The parrots will also engage in discussions and post "reviews" of the diet pills in order to make the diet pills seem more real while the egg accounts just inflate the like counts in order to make the mockingbirds as well as the parrots seem trustworthy. The egg accounts do not post any content, they just follow parrots.
\\
\\ The link provided by the mockingbirds seem real(with pictures of famous people). When a customer orders a free trial, they register the credit card and subsequently lose their money. 
\\
The point of this example is to show how easily for example likes can be misguiding for a company that tries to establish themselves as a brand within social media. 

\section{Do these metrics influence buyers?}\label{metricInfluence}
As we have seen, metrics can be obfuscated by many different factors; "click farms", bots, and other variables which can affect social media analytics. These factors may present in many different ways. In an article entitled "Beyond likes and tweets: Consumer engagement behavior and movie box office in social media", \cite{Oh2016}, they researched how consumer engagement behavior(CEB) was associated with economic performance based on popular movies released in the US. The study found that CEB in Facebook and YouTube correlated with gross-revenue in opening-week movies. The study further concluded that CEB played a pertinent role in relation to future economic performance, so in some cases pure metrics can have great effects on the consumer itself.
%%
%%
\\\\ Another good example of this is metrics in movies. In particular ratings on movies and Tv-series through IMDB(Internet movie data base), which is used to share users' opinions on films and tv-series in the form of ratings(top 250 movies, bottom 100) as well as recommendations and critics, both professional and user generated. In a paper entitled "Judgement devices and the evaluation of singularities: The use of performance ratings and narrative information to guide film viewer choice"(\cite{Bialecki2016}), the researchers noted that the use of metrics such as ratings(moviegoers set a threshold in which served as a "hurdle" in which movies they wanted to see had to pass). Research also indicated that if a movie had a high rating or were in the top 250 rating-list, it was likely moviegoers would see it because of the ratings themselves. Some subjects noted that the movie did not meet their expectations, but that they had to watch it because of the the ratings:
\begin{quotation}\label{MovieReview}
\textbf{\textit{"}}I rented this movie on the strength of the ratings and glowing
reviews at this site [IMDb]. “Brilliant”,they said. “Dark and beautiful”,
they wrote. 8.4 stars. Well, all I can say is, these people
must have been on some serious drugs when [they] saw this
totally inane movie. . .I give this movie 1 black hole.\textbf{\textit{"}}
\newline \mbox{} \hfill \cite{Bialecki2016}
\end{quotation}

\label{Emil:oogie}
As we can conclude from \cite{Bialecki2016}, we can say that people buy largely based on word of mouth as well as from reviews or "click metrics" (in the form of stars on IMDB). While doing research for this article, I found several websites which both offered botting specifically for IMDB(increase in stars), as well as a direct example of a movie in which had been a "bottom 200-list"-er for a long time, and was considered one of the biggest movieflops, gathering great nominations such as "Worst picture", "worst screen ensamble" \cite{emil:wiki:oogieacc} as well as a IMDB rating of 2.0 according to a petition \cite{Emil:change:Oogieloves}. The petition goes on to describe how the rating of the movie suddenly skyrocketed in June of 2013. in \ref{IMDB:rate} we can see the weighting of votes, most of which are at 10, which is the highest. This seems peculiar given the accolades which the movie was given and the initial score as well as the sudden boom. The ramifications of such actions will be discussed in the discussion part of this paper. 

\begin{figure}[h]\label{imdbratingCurve}
\centering
\includegraphics[scale=1]{fig/imdbstats.png}
\caption{IMDB rating of "The Oogieloves in the Big Balloon Adventure" in 2016, showing the skewed ratings.Courtesy of IMDB(\cite{IMDB:Oogieloves})}
\label{IMDB:rate}
\end{figure}

Another example of research done in this field is an article entitled "The Influence of Social Media: Twitter Usage Pattern during the 2014 Super Bowl Game" which analyzed how people used twitter during the superbowl event specifically looking at consumer interaction with advertisement.
The researchers used data mining in order to find correlations in data which may not be otherwise easily delectable. 
In the article(\cite{HyeonjeongShin2015}), the researchers asked two research questions: "How does the overall number of tweets differ between a game day and non-game day?"\cite{HyeonjeongShin2015}) and "What major topics in commercial related tweets were exchanged during the 2014 superbowl game?"\cite{HyeonjeongShin2015}). The data analysed for research question 1 (How does the overall number of tweets differ between a game day and non-game day?) indicated that Super bowl commercials created a buzz. The research also indicated that tweets about a commercial product accounted for 33\% of total commercial tweets generated on superbowl day(february 2) as opposed to 0.64\% and 9.48\% on january 26 and february 9th. Further, more than half of tweets(37 of 49 tweets) registered mentioned Budweiser products or their commercial. 
\\
\\
Even though these examples do not directly prove that consumers act on these type of data, one can draw the conclusion that user interacting might create a buzz and get your brand out there, thus leading to potentially more customers, subconsciously or otherwise. 

In the case of IMDB-ratings, they seem to influence what movies people want to see, and therefore revenue of movies which are featured in a popular way. This is backed up in a article entitled "An empirical investigation of user and system recommendations in e-commerce" about general statistics with e-commerce metrics \cite{Lin2014111} which could report that:
\begin{quote}
\begin{itemize}\label{userreccommendation}
\item {1\% increase in user recommendation volume increases product sales by 0.013\%.}
\item {1\% increase in user recommendation valence increases product sales by 0.022\%.}
\item {1\% increase in system recommendation strength increases product sales by 0.006\%.}
\end{itemize}
\end{quote}
\section{How can the statistics become irrelevant?}\label{Labl:emil:election}
Historically, social media in general has been said to increase globalization as well as further establishing democracy and free flow of information to areas which would otherwise not have this information available. But shortly after the 2016 U.S election, an article entitled "Social bots distort the 2016 U.S. Presidential election online discussion" \cite{Emil:FM7090} discussed how these types of communication tools are used to manipulated the online discussion. The study investigates how social media bots as well as algorithm driven entities appear as legitimate users and how they affect the political election in the 2016 U.S presidential election. In the article, the researchers uses "state-of-the-art" bot detection tools in order to uncover the user population that might not be human. The data collection for this article consisted of two roughly equal lists of hash tags in order to get the most equal coverage of both presidential candidates. \\ \\
The researchers queried the Twitter API every 10 seconds, contioniously and without interuption in three periods between 16th of september as well as 21th of october\cite{Emil:FM7090}. The data they got is listed in table 2 in \cite{Emil:FM7090} and tells us that from the $20,772,153$ tweets gathered, there were $2,782,418$ distinct users. 
\\
\\
For bot detection the researchers used a small Python script(BotorNot) which tested over a thousand features like 
\begin{quote}
"content, network structure, temporal activity, user profile data, and sentiment analysis to produce a score that suggests the likelihood that the inspected account is indeed a social bot."
\cite{Emil:FM7090}

\end{quote}
Further, they found that the two most important revealing factors were metadata as well as user statistics associated with the user account(s). \\
\\
The results are summarized in \cite{Emil:FM7090} table 3. Table 3 shows us statistics for the top $50,000$ accounts BotorNot gave a rating higher than the detection threshold. Table 3 shows us that from the $20,772,153$\cite{Emil:FM7090} tweets, the majority of tweets $10,303,251$ or (\textbf{$81.51\%$})\cite{Emil:FM7090} came from humans, but the remarkable result here is the $2,330,252$ or $18.45\%$ \cite{Emil:FM7090}which came from bots. The researchers used SentiStrength\cite{Emil:FM7090} to detect sentiment. SentiStrength gives a sentiment score, as well as effectively capture positive and negative emotions with up to 70.6\% and 72.8\% \cite{Emil:FM7090} accuracy. 


This allowed for the researchers to indicate the positiveness as well as negativeness of both candidates' tweets. The tweets from Trumps' bot supporters were almost all positive, and  according to SentiStrength were among the most positive of the entire data set, the researchers conclude that this is contradictory to the general negative tone towards which characterizes Trumps 2016 presidential campaign. The researchers further indicate that since the bots are categorically more positive, they can create the sense of a large grassroots following, which in reality are bots posing as humans. 
\\
\\
On the other side, Clinton's human supporters give on average more positive sentiments toward their candidate than what the bots does. The researchers found a more natural distribution of tweet sentiments in both groups which indicated that they were roughly equal both in terms of positive and negative tweets in the pro-Clinton discussion. 
\\
\\
On manual inspection, the researchers analyzed \#nevertrump and \#neverhillary and found that \#nevertrump got $105,906$ \cite{Emil:FM7090}positive tweets and $118,661$\cite{Emil:FM7090}negative tweets, which gave roughly a equal representation of both negative and positive numbers. On the other side, analyzing \#neverhillary found that this hashtag has significantly more negative tweets($204,418$)than positive ($171,877$) \cite{Emil:FM7090} giving a unequal distribution of numbers.
\\
\\

Concluding, the researchers found that there were about $400,000$ bots involved in the political discussion about the presidential election generating aboout $3.8$ million tweets.
Further, the issue of social bots engaged in political discussions poses three more issues. The first issue posed in the conclusion is that social bots may redistribute influence across suspicious accounts which may operate with malicious purposes. The second issue is that the political discussion might be shaped and even polarized by bots by shaping the discussion. The third issue posed is that these bots might spread misinformation as well as unverified information might be spread.
\\
\\
Further, they concluded that determining who was the source of the problem was difficult:
\begin{quote}\label{Emil:conclusion:ownershipofBots}
\textbf{"}it is impossible to determine who operates such bots. State- and non-state actors, local and foreign governments, political parties, private organizations, and even single individuals with adequate resources (Kollanyi, 2016), could obtain the operational capabilities and technical tools to deploy armies of social bots and affect the directions of online political conversation.\textbf{"} \cite{Emil:FM7090}
\end{quote}



\section{Bot detection}
With nearly 1.13 billion daily active users on Facebook \cite{FB:stats}, the importance of social media and how it influences people is important. As our lives gradually intertwine with social media and a personal digital presence, we become predisposed to marketing from friends and corporations.  But are we humans able to see the difference between what is user generated, or what is generated by bots or similar types of AI?
An article released in 2013, named "Is that a bot running the social media feed? Testing the differences in perceptions of communication quality for a human agent and a bot agent on Twitter"\cite{Edwards2014372} investigated the claim that humans would not see the difference between a bot, and a human in a single newsfeed on Twitter. They used a sample of 240 undergrad students enrolled in a communication course at a large mid-western research university with subjects ranging from the age of 18 to 39 years old. The researchers then used two treatment groups(the twitterbot and the human twitter agent). Those who chose to participate were given a link to a secure webpage with the study. After consenting to the study, the participants were randomly given one of the two twitter pages. The twitter profiles were designed to appear as information provided by the central for disease control(CDC) about sexually transmitted infections. The pages were identical all except for one important detail - the author; in the twitterbot case it was clearly stated that it was a CDC twitterbot, in the other case it was stated that the author was a CDC scientist. The study concluded that
 \begin{quote} \label{emil:runningbotfeed}
The findings demonstrate that Twitterbots can be viewed as credible, attractive,
competent in communication, and interactional, and might be an
appropriate program to transmit information in the social media
environment. 
\newline \mbox{} \hfill \cite{Edwards2014372}

\end{quote}
An article entitled "Globalization in social media and consumer relationships with brands in digital space" \cite{6959277120111201}, could conclude that 60\% of consumers\cite{6959277120111201} were strong brand advocates, and that they were more likely to buy the brand that they were advocates of. The paper continues to conclude that on a global basis, 18\% of users actively set up their online brand community\cite{6959277120111201}, giving techniques which are explained in \ref{obfuscatedmetrics} a bigger foothold in order to further obscure metrics, and causing marketeers and others who rely on this type of data to have problems.
\section{Discussion}
The effects social media bots can have on a society is huge. In this article we have seen several examples on how they affect us, and how their metrics impacts us. We have seen how bots can generate revenue through for example skewering ratings in a movie\ref{IMDB:rate}, to \textit{potentially} affecting a U.S election both in 2010\ref{midtermElection} and in 2016 \ref{Labl:emil:election}. With the recent election and its results, a valid line of discussion would be, did bots have any effect on this? In \ref{Labl:emil:election} we could see that researchers verbalized the problem in their conclusion, claiming that bots in a social media environment might shape and polarize the debate by for example "mucking the waters" with unverified data. In this article, we have read about how sales are affected by the user recommendations \ref{userreccommendation}. A discussion point could be if these types of metrics also influence voting. If these metrics influence voting, social media bots will have a great impact on voting, which in turn would allow for elections to be significantly shaped by who you follow and who you are friends with, but also who is in your media feed, which according to research presented in \ref{emil:runningbotfeed} shows that humans felt that Twitterbots \textbf{"}were credible, attractive, competent in communication as well as interactional in a social media environment\textbf{"} \cite{Edwards2014372}.
\\
\\
In other arenas than politics, we have seen how bots might have influenced IMDBs rating of a specific movie, resulting in a petition to improve bot detection in IMDB \ref{Emil:oogie}. The use of bots in this case, probably did not amount to much, but it might have gotten some extra revenue in which it was not entitled to if it bots were not present. The point of this example is to show how easy bots can influence humans. 
\\
\\
When it comes to detection of bots in metrics, there are various different ways to detect bots, but each of them has flaws, and it therefore cannot be guaranteed that bots do not influence metrics, without being detected. This allows for metrics to eventually slip under the radar, thus making bot generated metrics hard to isolate in a complex environment. This allows for companies to pad their own metrics, giving an image of success which may not be correct. 
\\
\\
If the bots that are used are not blatantly easy to detect, finding the bot-masters, as well as the customer might prove difficult. As concluded in \ref{Emil:conclusion:ownershipofBots}, the actor could be anyone with the appropriate means("State- and non-state actors, local and foreign governments, political parties, private organizations, and even single individuals with adequate resources").


\section{Conclusion}
In conclusion, we can say that this article presents many different aspects to this discussion about metrics. we could read that people were unable to distinguish between bots and humans, and in some cases actually preferred twitterbots, because they found them "credible, attractive, competent and interactional"\cite{Edwards2014372}.In \ref{metricInfluence} we could read about how consumers were basing their decision on what movie to watch by user recommendation and by rating alone, and also \ref{MovieReview} gave an example of how metrics could influence buyers, we could also find somewhat credible sources in \ref{imdbratingCurve} where these metrics were most likely skewed by bots. In the example of \ref{imdbratingCurve} the consequences are probably not that dire, but in \ref{Labl:emil:election} we presented the research around the 2016 Presidential election. In the article we could read that 400k bots might have affected the informationflow and may have shaped the discussion in popular twitterfeeds. The discussion discusses the consequences of fake metrics, and concludes that bot detection is complex, thus making my original hypothesis invalid because fake metrics in it self might be difficult to identify. 



\section{Grade defense}
In this report I've attempted to find background research in order to have a balanced discussion on whether or not bots have an impact on metrics. I believe the background information gathered is relevant, and that it gives a good understanding of how metrics influence the consumer today, and its actions. Since the available research about bot influencing metrics is scarce the article had to be written in a manner which gathered background information and then discussed the information found. I believe that this has been done in a sufficient manner. I have done my best to condense the material into what I believe gives a good reading experience, and I believe that my discussion holds a good standard. There are obvious improvements to be made, but they require that more material about bot metrics and their use is found, which I have been unable to. If I had found these types of articles, they would have made for a better balanced discussion because I would have had more data to conclude from. All in all I believe that my contributions should merit a B or C, given that I have written a discussion which appeared to be very relevant in todays society given the U.S 2016 election. 


