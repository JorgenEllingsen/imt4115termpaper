% !TEX encoding = UTF-8 Unicode
%!TEX root = thesis.tex
% !TEX spellcheck = en-US
%%=========================================
\chapter[Bot Metrics]{Does bots make click, text and behavior metrics irrelevant?}


\section*{What are bots and how are they used?}
In general, bots are types of software which run automated tasks(which are repetitive and allows for the program to be faster at doing some tasks than what normal humans would be able to. There are many types of bots, for example gaming bots which can be programmed to efficiently react faster than what a human would be able to on certain events in the game, thus allowing players to use time on something else. Others examples include auction bots, which hunt for bargains. Ebay went to court in 2000( \cite{Computerworld:Ebay}) in order to stop this type of behavior, the federal courts in turn then decided to block Bidders Edge and their bots from accessing Ebays API. 
\\
\\
Furthermore bots are used to mimic human activity, in for example service applications which appear to be human interaction but is instead a bot which for example tries to "help" you based on keywords. Bots are used in many areas, but are probably most known for their maliciousness in areas such as spam, bandwidth-thiefs, scrapers, worms and viruses as well as as nodes in a larger scale bot-nets which in tern is used for example for DDOS. These bots can be part of bigger bot-nets in order to for example generate revenue through for example click fraud. 
%%
\newpage 
%%
In \cite{Observer:FakeTRAF} the problem and its extent is very well described: \begin{quote}
A recent study by comScore found that 54 percent of display ads shown in thousands of campaigns between May 2012 and February 2013 never appeared in front of a human being. Rather, the traffic came from bots. As an advertiser, this would be like buying a billboard you were told was seen by thousands of cars a day only to find out that was because the billboard sat next to the assembly line at a Ford plant. Sure, that’s a lot of cars, but there’s no one in them.
\\
\\
In fact, the system is so broken that, for some publishers, knowingly buying traffic that comes from bots is part of their business model. An anonymous publishing executive, who claimed to be buying up to \$35,000 worth of traffic per day, recently told Digiday that for publishers running an arbitrage model, all that matters is profit; quality of traffic does not factor into the equation.
\newline \mbox{} \hfill \citet{Observer:FakeTRAF}
\end{quote}
This quote from comScore highlights a important point in todays society. How does click metrics influence us, and how are they used? Are they useless, can they be used to measure something even though results most likely have been contaminated? and how does this type of behavior influence the buyers decision? These and some others are the question I will answer during this chapter. 
