% !TEX encoding = UTF-8 Unicode
%!TEX root = thesis.tex
% !TEX spellcheck = en-US
%%=========================================
\newpage
\chapter{Approaches for Detecting Robots \\ in Social Media}
\section*{Management Summary}


\section{Introduction}
Software robots are often called bot.

\section{Definition and History of Social Bots} 
This section will introduce the term social bot formally and give a short overview about the history of this topic.

In order to be able to discuss social media bot detection, we need a clear understanding of what social bots actually are.  For that, we use the definition given by Ferrara et al. in their article The Rise of Social Bots:
\begin{quote}
	"A social bot is a computer algorithm that automatically produces content and interacts with humans on social media, trying to emulate and possibly alter their behavior." \cite{ferrara15}
\end{quote}

The root of of social bots, or just bots how we will sometimes call them here as well, can probably be found in the Turing test, developed by Alan Turing in 1950 \cite{turing}. It involves three parties, two of which are human and one is a computer program. While one human is having a conversation with the software, it is the task of the other human to identify the program. If he is not able to do so, the software is passing the Turing test. This led to the development of a lot of so called chatbots, which just aimed to appear as human as possible in a conversation.  

A rather famous and often cited example for such a chatbot is ELIZA, introduced by Joseph Weizenbaum 1966 in \cite{eliza}. It mimicked a psychotherapist and showed \mbox{that -- at} least some kind\mbox{ of -- communication} between a human and a computer is possible.

Since then, a lot of things have changed. Today, bots are a lot more than bare entertainment or proof of concept. With the rise of social networks like Facebook and Twitter the possibilities for social bots increased dramatically.  As we will see in the next section, nowadays bots are used to spread messages, for marketing, to generate clicks and for a lot more.  %%IST DAS WIRKLICH SO? CHECKEN WENN SECTION GESCHRIEBEN!!


\section{Why is Bot Detection in Social Media Important?}%Better "Bot-Detection"?
\begin{itemize}
	\item Information flood -> need to get the message through
	\item influence political mood
	\item marketing
\end{itemize}

\section{Social Media Bot Detection Approaches}


\subsection{Based on Social Network Information}


\subsection{Based on Crowd-Sourcing}


\subsection{Based on Machine Learning Methods}


\section{Summary and Outlook}


























\newpage