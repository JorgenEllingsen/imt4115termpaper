% !TEX encoding = UTF-8 Unicode
%!TEX root = thesis.tex
% !TEX spellcheck = en-US
%%=========================================
\newpage
\chapter{Approaches for Detecting Robots \\ in Social Media}
%% NOCH EIN KAPITEL: "VIEWED SOCIAL MEDIA PLATTFORMS"? -> erkläre hier betrachtete social media plattformen kurz damit man später weiß wovon man redet?


\section*{Management Summary}
This is the management summary blablabla mhmhm...

\section{Introduction}
Software robots are often called bot. <- muss hier irgendwie rein 

\section{Definition and History of Social Bots} 
This section will introduce the term social bot formally and give a short overview about the beginning and the development of this topic.

In order to be able to discuss social media bot detection, we need a clear understanding of what social bots actually are.  For that, we use the definition given by Ferrara et al. in their article The Rise of Social Bots:
\begin{quote}
	"A social bot is a computer algorithm that automatically produces content and interacts with humans on social media, trying to emulate and possibly alter their behavior." \cite{ferrara15}
\end{quote}

The root of of social bots, or just bots how we will sometimes call them here as well, can probably be found in the Turing test, developed by Alan Turing in 1950 \cite{turing}. It involves three parties, two of which are human and one is a computer program. While one human is having a conversation with the software, it is the task of the other human to identify the program. If he is not able to do so, the software is passing the Turing test. This led to the development of a lot of so called chatbots, which just aimed to appear as human as possible in a conversation.  

A rather famous and often cited example for such a chatbot is ELIZA, introduced by Joseph Weizenbaum 1966 in \cite{eliza}. It mimicked a psychotherapist and showed \mbox{that -- at} least some kind\\ \mbox{ of -- communication} between a human and a computer is possible.

Since then, a lot of things have changed. Today, bots are a lot more than bare entertainment or proof of concept. With the triumph of the Internet and especially social networks like Facebook and Twitter, the possible use cases for social bots have increased dramatically. While they were initially mostly used to simply post content, today they are able to credibly interact with each other and even humans \cite{boshmaf13, hwang12}. As we will see in the next section, nowadays bots are used to spread messages, for marketing and a lot more.


\section{Why is Bot Detection in Social Media Important?}
\begin{itemize}
	\item social media sometimes substitutes personal contact
	\item Information flood -> need to get the message through
	\item influence political mood (smoke screening in Dissecting a Social Botnet: Growth, Content	and Influence in Twitter), astroturfing
	\item marketing
	\item false information (boston marathon: cassa 2013)
	\item seem fame
	\item stock exch..
	\item https://sysomos.com/inside-twitter/most-active-twitter-user-data 32\% of tweets by bots!
	\item harvest publicly inavailable data (through friendship you can see more!) (voices from the fronts..)
\end{itemize}
---> section "engineered social tampering" and following in the rise of social bots!! 
\textbf{Im Prinzip angriffsübersicht?}

+ Key Challenges in Defending Against Malicious Socialbots \cite{boshmaf12}
\section{Social Bot Detection Approaches}
In this chapter we want to introduce several techniques for detecting social bots. Based on Ferrara et al. \cite{ferrara15} we distinguish three detection approach classes. 

The first category of detection approaches is based on social network information. They are also called graph-based, since they map users and their relations into a graph and then try to identify bots in the hereby obtained social network by means of graph theory. 

Afterwards we will discuss crowd-sourcing based social bot detection approaches. They use actual humans to detect bots, assuming that the human ability to notice details in communication will make this an easy task.

The last category we want to elaborate on are detection approaches based on behavioral features. Mechanisms that make use of this approach try to observe behavioral patterns that are typical for social bots. By doing so they aim to distinguish bots from human users. \cite{ferrara15}. %%IS THIS CITATION NEEDED? ALREADY SAID IN FIRST PARAGRAPH(BASED ON...)


In the following sub sections, we will go into detail about each of these three approaches and illustrate them using real detection systems. It has to be noted though, that the borders between the individual approaches is not always very clear, so that some examples could also be mentioned in a different category. Many schemes for example, use some kind of graph theory. However, we try to categorize them by their core traits.

\subsection{Based on Social Network Information}
A term that is often used in combination with detection of bots by using social networks is sybil or the sybil attack. It was presented as a thread to distributed systems by John R. Douceur in \cite{sybil}. In the specific context of social media platforms when conducting a sybil attack, an attacker creates a large amount of fake identities in a system to the point where these identities make up a considerable fraction of the system's whole user base. When this is achieved, the attacker can influence the whole system and control its contents to a certain degree. A sybil, sybil node or sybil account is therefor simply one of the fake entities, or, depending on the attack architecture, just a social bot. It is not hard to see that social bot detection can, more specifically, be viewed as a defense against the sybil attack.   

The general proceeding of social network based bot detection approaches is rather simple. They map the users base of the social platform they aim to defend into a social graph, where a node is corresponding to a user and an edge between two nodes exists if there is a specific kind of relationships between the two respective users on the platform. The nodes can be hereby be distinguished in sybil nodes, respectively bots, and non-sybil nodes, respectively legitimate users. The goal of the detection approach is now, to identify whether a given node is a sybil or not \cite{comparison}.

There are a number of proposed social network based sybil detection schemes like for example SybilGuard \cite{sybilguard}, SybilInfer \cite{sybilinfer} or SumUp \cite{sumup}. While they all have different assumptions and use varying algorithms to achiever their goal, Viswanath et al. show in \cite{comparison}, that, at a high level, all of them work by them same principles.

Basically they can be viewed as graph partitioning algorithms, which partition a given graph into multiple disjoint subgraphs. As already mentioned above, ideally two subgraphs are assembled, one that contains only sybil nodes and one that contains only non-sybil nodes. Since a clear distinction is often hard to make, the approaches basically assign a rank to each node and decide afterwards, depending on several parameters, which ranks are classified as sybil and which as non-sybil. It is thereby obvious, that the ranking algorithm is crucial for the whole scheme. Though, of course, the ranking algorithms for the different detection schemes are differing, they generally have in common, that they base their rating on how tightly connected the respective node is to a known trusted node. Thus, they work by detecting local communities of nodes. In other words closely connected groups of nodes \cite{comparison}.

It is not hard to see, that these algorithms are therefor easy to deceive. If an attacker is able to establish so called attack edges, connections between his sybil nodes and non-sybil nodes which are connected to a trusted community, it gets significantly harder to identify the bots. A common assumption is that these attack edges are hard to create \cite{sybilguard}, which means, that legitimate users tend not to establish social network connections to social bots. However, Boshmaf et al. show in \cite{boshmaf11} that this assumption is to be questioned. 
They tried to infiltrate the social media platform Facebook with a large number of sybil accounts and tried to establish connections to real users. The average acceptance rate of their relationship requests came to about 20\% and could be increased to 80\% depending on how many indirect relationships between the sybil and the user already existed \cite{boshmaf11}. %%FACEBOOK / TWITTER etc. zitieren?

A well known example for a bot detection approach that is based on social network information is the Facebook Immune System (IMS) \cite{fis}. This system aims to defend the social media platform Facebook and its users not only against sybil attacks but also to prevent spam, malware distribution, phishing and so on.  To achieve this goal, it takes a lot more actions than the above described general approach for social network based detection schemes. The IMS runs checks on every action performed on Facebook in realtime in order to give attackers as less time as possible to accomplish there goals and to react. It classifies these actions according to predefined policies and makes it thereby possible to judge them and the corresponding users.

An example for this could be a newly created Facebook account, that sends a lot of friendship requests. These are used in Facebook to establish relationships between users. A legitimate user starts sending friendship requests usually mainly to people he knows and vice versa, that is, people that are likely to accept his friendship request. If a lot of these requests are now declined, this could be a indication for the system that the sending user might not be legitimate.  The IMS also makes heavy use of machine learning, to which we will come back in later on, and generates training data automatically in order to adapt to the fast changing attacks and user behavior \cite{fis}.

\subsection{Based on Crowd-Sourcing}
A rather straight forward approach for social bot detection is based on crowd-sourcing. In contrast to the above described social network information based approaches, a connection between bots and legitimate users is not a problem for schemes based on this approach -- at least not a direct one. The basic idea of crowd-sourcing based social bot detection is, to engage actual human workers to study user profiles and subsequently to decide whether it belongs to an actual user or a sybil.

In \cite{wangcrowd}, Wang et al. presented a study on the effectiveness of this approach and introduce a sample for a crowd-sourcing based social bot detection system.  For their tests they use sample data from Renren, China's most popular social media platform, Facebook US and Facebook India. They subdivide their human investigators, or simply testers, in expert workers, and crowd-sourced workers.  Each of the testers is presented a number of social media profiles and has to decide, whether it is a real one, or a sybil. While the experts achieve a detection rate at about 90\%, the crowd-sourced workers perform not as good individually. If their single votes are aggregated and used for a majority decision though, the results can significantly increased. If this is done for the expert workers, their results can be increased even more, too. A very desirable result of their studies is also that the false positive rate, that is the amount of profiles that are falsely classified as sybils, is in all groups very close to zero. This ensures, that the probability that legitimate users are accused to be social bots, what will probably offend them, is very low.

Wang et al. conclude, that it is very hard for sybil creators, to assemble social bots respectively profiles that are able to pass a "social turing test" and that crowd-sourcing based approaches can perform very well. They proceed with introducing a general practical system that is illustrated in figure \ref{crowdsys} \footnote{Turkers are in this case the former mentioned crowd-sourced workers.}.

\begin{figure}
	\centering
	\includegraphics[scale=0.4]{fig/crowdsys}
	\caption{A crowd-sourced sybil detection system introduced by Wang et al. \cite{wangcrowd}.}
	\label{crowdsys}
\end{figure}

The system is working like this: The social network respectively social media platform that is to be defended generates suspicious profiles. This can either happen through explicit reports from users or through automatically applied filters that detect abnormal behavior. The so obtained profiles are then checked by crowd-sourced workers, which are subdivided in accurate and very accurate. To establish this differentiation, some of the suspicious profiles are mixed with some profiles that are known to be sybils. They are presented to all workers and their results allow a classification and to filter out unreliable workers. In the actual checking process, the profiles are first presented to the accurate workers which make a majority vote. If their decision is not clear, the profiles are presented to the very accurate workers which once again make a majority vote. 
The authors claim, that a system like this is very reliable and cost effective. 

Though such a social bot detection approach based on crowd-sourcing seems very promising at first sight, several problems become obvious on closer examination. First of all, this approach works optimal if every newly created profile can be reviewed in the above described way. However, young social media platforms will probably not be able or willing to pay for this bot detection scheme. A factor that may influence this could be also, that those platforms probably do not have huge issues with the social bot problem in their early stages. Once the platform has grown an the issue arises though, their is usually already a huge user base that has to be reviewed retrospectively. This is not optimal and may take a considerable amount of time.

Another point that mus be taken into consideration is that this approach might not be suitable for all types of social media platforms. While it might be well suited for platforms like Facebook or Renren, where profile pages can be customized a lot and contain usually plenty of information, other networks, like for example Twitter could be less appropriate. Generally it can be said that this detection approach is highly based on the information given in the users' profiles. If the information contained in the profiles is low, human investigators will probably not be able to distinguish as precisely as in the tests of Wang et al.

The last and probably most difficult issue that has to be mentioned is connected to this. Human investigators that are charged with distinguishing between bots and legitimate users have to have access to the profiles of the users. While those profiles are sometimes publicly available anyways, detailed profiles, which are more interesting for this approach, are, as discussed above, usually only visible for certain users. It is not hard to see the privacy issue that arises here, especially when keeping in mind, that the investigators are -- at least to a great extent -- only crowd-sourced workers, that can not be supervised as easily as ordinary employees \cite{ferrara15}.


\subsection{Based on Features}
The last detection approach for social bots is based on features. Features hereby means observable characteristics of posted information, profiles and possibly everything else that can be associated to the account in question. Schemes that make use of this approach often also use machine learning algorithms. That are algorithms that are able to make decisions based on learned patterns. They basically consist of two phases. In the first phase, the so called training phase, the algorithm processes training data which is labeled with the decision that should be made on this piece of data. By processing a lot of training data, the algorithm learns the features that are indicators for a certain decision. In the second phase, real data is processed and the algorithm is able to make decisions based on the learned data. Thus, schemes that make use of this detection approach can be compared to anomaly based intrusion detection systems. In the social bot detection topic, the decision being made is usually whether a specific account is controlled by a bot or a human. Since it is not easy to make a clear decision in many cases, there are often multiple decision options that each express a degree of certainty.

A well known example for this kind of social bot detection is the Bot or Not? system, presented by Davis et al. in \cite{botornot}. It is a publicly available\footnote{http://truthy.indiana.edu/botornot} system that allows a feature based social bot detection for accounts on the social media platform Twitter. Bot or Not? was trained with about 31.000 account samples of both, human and social bot accounts. For classification makes use of more than 1.000 features that can be assigned to six feature-classes. In the following we want to mention these classes shortly:
\begin{itemize}
	\item \textbf{Network} features reflect on the spreading of information, for example citations of different users or the like.
	\item \textbf{User} features take the information given in the actual account into consideration. Meta-data, that is basically data about data, is especially relevant in this feature. 
	\item \textbf{Friends} features are about the social relationships the account in question has.
	\item \textbf{Temporal} features analyze issues like the rate in which content is produced and the like.
	\item \textbf{Content} features are about wording characteristics in the texts produced by the relevant account.
	\item \textbf{Sentiment} features reflect on emotional aspects and are obtained by using specific algorithms \cite{botornot}.
\end{itemize}  
If a Twitter account name is entered in the Bot or Not? system, it checks all of those features and afterwards decides, based on the former mentioned training data, whether this account is a human or bot account.

Another social bot detection approach that can be classified as, kind of, feature based relies on detecting the coordinated behavior between social bots. Schemes that make use of this approach focus generally on the above described network, content and temporal features and often incorporate some graph theory as well. They usually aim at detecting attacks which try to distract from unpleasant facts or to distribute fake facts, like in the above described astroturf attack.  An example for such a scheme is SynchroTrap, presented by Cao et al. in \cite{synchrotrap}. Again, comparable to anomaly based intrusion detection systems, SynchroTrap makes use of clustering algorithms. These algorithms try to group values, that are similar in some aspects into distinct sets. 

The system monitors all user activity on a social media platform over an extended time period and monitors aggregated user actions. Then, the pairwise similarity between actions is determined and a hierarchical clustering algorithm is used to group users that show similar behavior at approximately the same time. If a cluster is growing too big, it can be assumed as malicious, since legitimate users tend to behave diverse \cite{synchrotrap}.

An issue with this detection method is though, that it is for example not able to detect bot accounts that show a mixture of bot and human behavior. Thus, the detection systems have to be developed further in order to keep up with the fast evolving social bots and their evermore sophisticated behavior \cite{ferrara15}.


\section{Summary and Outlook}

short summary of everything we discussed above

combination of approaches best

social bots are not necessarily evil


























\newpage